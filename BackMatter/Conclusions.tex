%===================================================================================
% Chapter: Conclusions
%===================================================================================
\chapter*{Conclusiones}\label{chapter:conclusions}
\addcontentsline{toc}{chapter}{Conclusiones}
%===================================================================================

En este trabajo se propuso y comprobó la eficiencia de un sistema de detección y caracterización de lanzamientos de un pitcher de béisbol utilizando una cámara PlayStation Eye. Para ello se hizo un recorrido por la teoría que gobierna cada una de las etapas del sistema implementado. Se propuso una configuración para la cámara y una posición en el terreno para la misma de acuerdo con las características que posee un lanzamiento en el béisbol. Se estudiaron las características geométricas del \textit{Home Plate} para su posterior detección en la imagen y se propusieron un conjunto de filtros para descartar falsos resultados en esta detección. En el análisis de la detección de la pelota se propuso un algoritmo de BS que brindara buenos resultados en el ambiente en que se utilizará este sistema y se construyeron un conjunto de pruebas para comprobar la eficiencia de cada etapa en distintas condiciones. Para la aproximación de la trayectoria de la pelota se hizo uso de \textit{RANSAC}, un método iterativo que permite calcular los parámetros de un modelo matemático de un conjunto de datos observados que contiene \textit{outliers}.


El resultado principal de este trabajo es la propuesta de un sistema que permite determinar si un lanzamiento de un pitcher de béisbol fue strike o bola con un costo inferior al de las soluciones comerciales existentes, con el objetivo de continuar trabajando en el mejoramiento de los lanzadores, en aras de que el pasatiempo nacional cubano vuelva a alcanzar el nivel mundial que siempre a tenido.

