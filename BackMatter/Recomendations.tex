%===================================================================================
% Chapter: Recomendations
%===================================================================================
\chapter*{Recomendaciones}\label{chapter:Recomendations}
\addcontentsline{toc}{chapter}{Recomendaciones}
%===================================================================================

El sistema es perfectamente ajustable para su funcionamiento utilizando una cámara estereo, el problema consiste en el alto costo de estos dispositivos. Sería deseable comparar el uso de una cámara estereo en este sistema con el funcionamiento de la propuesta presentada en este trabajo. Una propuesta de bajo costo para un sistema estereoscópico se puede encontrar en \cite{DVD}.

Una de las funcionalidades que pudiera brindar el sistema sería información estadística con respecto al comportamiento de los lanzadores luego de un conjunto de lanzamientos y conocer con una mayor presición sobre que zona de strike estos realizán sus lanzamientos con mayor frecuencia. Esto permitiría conocer si un lanzador trabaja sobre la zona de \textit{duda} (bordes de la zona de strike) o sobre el centro de la zona de strike.

Para facilitar el uso de este sistema se propone desarrollar una aplicación visual que facilite el proceso de calibración y permita visualizar la trayectoria del lanzamiento en conjunto con el lugar de la zona de strike por donde paso.
