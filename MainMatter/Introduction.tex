%===================================================================================
% Chapter: Introduction
%===================================================================================
\chapter*{Introducción}\label{chapter:introduction}
\addcontentsline{toc}{chapter}{Introducción}
%===================================================================================

Se dice que los deportes son el pegamento social de la sociedad. Permite a las
personas interactuar independientemente de su estatus social, edad, etc. Con el
auge de los medios de comunicación, se ha canalizado una cantidad significativa
de recursos hacia los deportes para mejorar la comprensión, el rendimiento y la
presentación. Por ejemplo, áreas como la evaluación del rendimiento, que antes
eran principalmente de interés para entrenadores y científicos deportivos, ahora
encuentran aplicaciones en medios de difusión y otros medios, impulsadas por el
uso cada vez mayor de la visualización de deportes en línea, que proporciona una
forma de realizar todo tipo de estadísticas de actuaciones disponibles para los
espectadores. La visión por computadora ha comenzado recientemente a desempeñar
un papel importante en los deportes, como se ve, por ejemplo, en el fútbol, donde
los gráficos basados en la visión por computadora en tiempo real mejoran diferentes
aspectos del juego.

Los algoritmos de visión por computadora tienen un gran potencial en muchos aspectos
de los deportes, que van desde la anotación automática de secuencias retransmitidas,
hasta una mejor comprensión de las lesiones deportivas y una mejor visualización.
Hasta ahora, el uso de la visión por computadora en los deportes se ha diseminado
entre diferentes disciplinas.

En el mundo de los deportes se ha tratado continuamente de desarrollar técnicas
de perfeccionamiento, tanto para satisfacer necesidades de los jugadores como
de los árbitros que en estos participan. Existen varios acercamientos a la hora de
desarrollar estas técnicas, por ejemplo, en el campo tecnológico se han
desarrollado diversas variantes como la implementación de accesorios o la de
sistemas computacionales. Dentro de los sistemas computacionales se han
realizado varios intentos de mejorar la precisión con que los árbitros inciden en el
juego, para así hacer un poco más de justicia por decirlo de alguna forma.

Recientemente se han lanzado al mercado varios productos para mejorar la
precisión de los árbitros. Tanto es así que en varios deportes en la actualidad ya
se hace uso de estas tecnologías, como es el caso del béisbol, el fútbol, el tenis,
el voleibol, el fútbol americano o fútbol Rugby, entre otros que se han ido sumando
a esta lista.

\begin{figure}[H]
	\centering
	\begin{subfigure}[b]{0.49\linewidth}
		\includegraphics[width=\linewidth]{Graphics/PitchFX_1.jpg}
		\caption{}
	\end{subfigure}
	\begin{subfigure}[b]{0.49\linewidth}
		\includegraphics[width=\linewidth]{Graphics/HawkEye_1.jpg}
		\caption{}
	\end{subfigure}
    \begin{subfigure}[b]{0.49\linewidth}
        \includegraphics[width=\linewidth]{Graphics/SportLive_1.jpg}
        \caption{}
    \end{subfigure}
    \begin{subfigure}[b]{0.49\linewidth}
        \includegraphics[width=\linewidth, height=3.5cm]{Graphics/K-Zone_1.jpg}
        \caption{}
    \end{subfigure}
    \caption{Tecnologías más usadas: (a) PITCHf/x, (b) Ojo de Halcón, (c) Sport Live, (d) K-Zone.}
	\label{fig:technologys}
\end{figure}

Estos sistemas usan entre 7 y 10 cámaras conectadas entre sí para poder obturar
en el mismo instante de tiempo, esto produce una alta eficiencia y una gran
precisión a la hora de ubicar en el espacio a cualquier objeto que en este se
encuentre. Cada una de estas cámaras son de una alta calidad por lo que tienen
un elevado precio en el mercado que oscila entre 3000 y 5000usd. Todo esto hace
que instalar estos sistemas en un estadio sea extremadamente costoso con
montos que superan los 50 000usd. Existe otro pequeño grupo de cámaras de
una calidad considerable pero que, sin embargo, son de muy bajo costo, por
ejemplo, la cámara PlayStation Eye.

\begin{figure}[H]
    \centering
    \includegraphics{Graphics/PlayStation-Eye.jpg}
    \caption{Cámara PlayStation Eye.}
\end{figure}

La cámara PlayStation Eye se lanzó al mercado en sus inicios con un precio de
20usd, en estos momentos ese precio es de 5usd, esta cámara es capaz de
capturar hasta 120 cuadros por segundo, a diferencia de las cámaras web
regulares que solo captan 30 cuadros por segundo. Además, cuenta con un
chipset de alta calidad comparable con los encontrados en muchas cámaras
dedicadas a la visión por computadora. Esto combinado con su bajo costo la
convierte en una gran opción a la hora de tomar en cuenta cual dispositivo utilizar
para realizar un sistema computacional basado en visión por computadora.
Debido a lo anterior nos hemos planteado si será posible determinar en tiempo
real y con un error de precisión mínimo si un lanzamiento de un pitcher de béisbol
fue strike o bola utilizando una sola cámara PlayStation Eye. Para esto se diseñará
una herramienta que nos permita colocar dicha cámara a tres metros encima del
Home Plate para poder determinar con mayor exactitud la trayectoria de la pelota.

\subsection*{Objetivos}

Atendiendo al alto costo de los sistemas ya existentes y los limitados recursos que
posee nuestro país, se propone como objetivo principal desarrollar una aplicación
que permita determinar si un lanzamiento de un pitcher de béisbol
fue strike o bola mediante técnicas de visión por computadora utilizando una única
cámara PlayStation Eye.

A continuación, se enumeran los siguientes objetivos específicos:
\begin{itemize}
    \item {Probar técnicas para la eliminación de ruido en tiempo real dada una secuencia de
            imagenes capturadas por la cámara.}
    \item {Determinar si en una imagen se encuentra el Home Plate.}
    \item {Ubicar los 5 puntos del Home Plate en la imagen.}
    \item {Transformar la imagen brindada por la cámara a una vista top view.}
    \item {Determinar si en una imagen se encuentra la pelota.}
    \item {Determinar la trayectoria de la pelota en una secuencia de imagenes.}
    \item {Determinar si el lanzamiento fue strike o no, así como la velocidad del lanzamiento.}
    \item {Obtener datos estadísticos que permitan determinar la precisión y la
            velocidad de un pitcher después de un conjunto de lanzamientos.}
\end{itemize}

\subsection*{Contribuciones}

\section*{Organización de la Tesis}

% El capítulo ...