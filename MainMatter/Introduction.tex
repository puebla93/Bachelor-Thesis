%===================================================================================
% Chapter: Introduction
%===================================================================================
\chapter*{Introducción}\label{chapter:introduction}
\addcontentsline{toc}{chapter}{Introducción}
%===================================================================================

Se dice que los deportes son el pegamento social de la sociedad. Permite a las personas interactuar independientemente de su estatus social, edad, etc. Con el auge de los medios de comunicación, se ha canalizado una cantidad significativa de recursos hacia los deportes para mejorar su comprensión, rendimiento y presentación. Por ejemplo, áreas como la evaluación del rendimiento, que antes eran principalmente de interés para entrenadores y científicos deportivos, ahora encuentran aplicaciones en medios de difusión y otros medios, impulsadas por el uso cada vez mayor de la visualización de deportes en vivo. La visión por computadora ha comenzado recientemente a desempeñar un papel importante en los deportes, como se ve, por ejemplo, en el fútbol, donde los gráficos basados en la visión por computadora en tiempo real mejoran diferentes aspectos del juego.

Los algoritmos de visión por computadora tienen un gran potencial en muchos aspectos de los deportes, que van desde la anotación automática de secuencias retransmitidas, hasta una mejor comprensión de las lesiones deportivas y una mejor visualización. Hasta ahora, el uso de la visión por computadora en los deportes se ha diseminado entre diferentes disciplinas.

Hace ya algún tiempo se vienen abriendo camino en el mundo deportivo varias tecnologías que en tiempo real mejoran diferentes aspectos del juego. Tanto es así que en varios deportes en la actualidad ya se hace uso de dichas tecnologías, como es el caso del béisbol, el fútbol, el tenis, el voleibol, el fútbol americano o fútbol Rugby, entre otros que se han ido sumando a esta lista (ver Fig. \ref{fig:technologys}).

\begin{figure}[h!]
	\centering
	\begin{subfigure}[b]{0.49\linewidth}
        \centering
		\includegraphics[width=\linewidth]{Graphics/PitchFX_1.jpg}
		\caption{}
	\end{subfigure}
	\begin{subfigure}[b]{0.49\linewidth}
        \centering
		\includegraphics[width=\linewidth]{Graphics/HawkEye_1.jpg}
		\caption{}
	\end{subfigure}
    \begin{subfigure}[b]{0.49\linewidth}
        \centering
        \includegraphics[width=\linewidth]{Graphics/SportLive_1.jpg}
        \caption{}
    \end{subfigure}
    \begin{subfigure}[b]{0.49\linewidth}
        \centering
        \includegraphics[width=\linewidth, height=3.5cm]{Graphics/K-Zone_1.jpg}
        \caption{}
    \end{subfigure}
    \caption{Tecnologías más usadas en la actualidad : (a) PITCHf/x, (b) Ojo de Halcón,
            (c) Sport Live, (d) K-Zone.}
	\label{fig:technologys}
\end{figure}

Unos de los deportes en donde mayor aceptación han tenido estas tecnologías es el béisbol, sobresaliendo \textit{PITCHf/x} \cite{Pitchf/x}. \textit{PITCHf/x}, creado y mantenido por Sportvision \cite{Sportvision}, consiste en un sistema que rastrea y registra digitalmente en vivo la trayectoria completa de los lanzamientos de béisbol, con una pulgada de precisión, lo que permite nuevas formas de entretenimiento y análisis de béisbol para ligas, equipos, emisoras y fanáticos. Este sistema hizo su debut en los playoffs de las Grandes Ligas de Béisbol \cite[MLB]{MLB} de 2006. La tecnología rica en datos es el resultado de más de siete años y millones de dólares en investigación y desarrollo. Los datos del sistema a menudo son utilizados por la televisión para mostrar una representación visual de los lanzamientos y si un lanzamiento entró en la zona de strike. \textit{PITCHf/x} también se usa para determinar el tipo de lanzamiento realizado, como una bola rápida(recta), curva o slider. La MLB usa los datos de \textit{PITCHf/x} en su Sistema de Evaluación de Zona, usándolo para calificar y proporcionar retroalimentación a los árbitros. Los analistas de sabermetría observan que la precisión de los árbitros ha mejorado después de que se introdujo la tecnología en la MLB \cite{SlateSportArticles}.

El sistema se ha instalado en los 30 estadios de la MLB, y actualmente rastrea lanzamientos para cada juego de la misma. Utiliza tres cámaras de seguimiento y un sistema central de seguimiento del lanzamiento en cada estadio. Cada cámara de seguimiento registra el lanzamiento desde el momento en que la pelota sale de la mano del lanzador hasta que cruza el plato, y luego envía esta información al sistema de seguimiento para calcular y almacenar el registro digital del lanzamiento, incluida la velocidad, ubicación y trayectoria. Los datos del lanzamiento pueden transmitirse en tiempo real para efectos de difusión de televisión, aplicaciones de consumidor, análisis de rendimiento u otras formas de entretenimiento y/o evaluación \cite{PITCHf/xSportvision}.

Para la temporada de 2017, \textit{PITCHf/x} fue rechazado y remplazado por \textit{TrackMan}, un componente de la plataforma de la MLB \textit{Statcast} \cite{Statcast}. \textit{Statcast} es una tecnología de seguimiento vanguardia que es capaz de medir aspectos del juego que antes no se podían cuantificar. Comenzando con el pitcher, \textit{Statcast} puede medir puntos de datos simples como la velocidad, al mismo tiempo que excava mucho más profundo, puesto que también mide el punto de liberación y la velocidad de rotación de cada lanzamiento. La plataforma también es capaz de medir la velocidad de salida, el ángulo de lanzamiento y el vector de la pelota cuando sale del bate. Desde allí también se puede rastrear el tiempo de suspensión y la distancia proyectada que recorre una pelota.

\textit{Statcast} tiene capacidades aún más amplias cuando se trata de rastrear corredores de bases y jugadores defensivos. \textit{Sprint Speed} \cite{SprintSpeed} es una medida que busca cuantificar con mayor precisión la velocidad, midiendo la cantidad de pies por segundo que corre un jugador, lo que ayuda a descomponer los factores que entran en juego en una base robada o un fildeo difícil. Además, el tiempo transcurrido desde el momento en que el lanzamiento del pitcher golpea el guante del catcher, hasta el momento en que el fildeador recibe el lanzamiento del catcher en el centro de la base, es rastreado en cada intento de robo. Entre otras cosas, \textit{Statcast} también puede monitorear qué tan lejos o eficientemente viajó un jugador en una jugada determinada, qué tan rápido se deshizo de la pelota un fildeador y la velocidad del lanzamiento resultante.

Dichos datos también han permitido la creación de métricas, como probabilidad de conectar un hit y probabilidad de capturarlo, diseñadas para analizar aspectos del juego que serían imposibles de medir con precisión sin \textit{Statcast}. Esto, a su vez, ha conducido a nuevas formas de evaluar los desempeños de los jugadores y proyectar con mayor precisión los resultados futuros, incluido el promedio(AVG) esperado de bateo (xBA), el porcentaje esperado de slugging (xSLG) y el promedio ponderado esperado en base (xwOBA)

Con tanta información ahora disponible para cuantificar lo previamente incuantificable, el potencial para una mayor innovación es grandioso. Nuevas métricas se seguirán presentando cada año para mejorar y avanzar en cómo se ve el juego. Gracias a \textit{Statcast}, el futuro del análisis de béisbol ha llegado oficialmente y te sentirás íntimamente aclimatado con cada una de las funciones del mismo para cambiar tus juegos.

Es válido señalar que todas estas tecnologías tienen un elevadísimo costo puesto que usan cámaras y radares de muy alta calidad, todo esto con el fin de obtener una mayor precisión en sus datos y estadísticas.

En Cuba el deporte nacional es el béisbol y hace varios años que se viene sañalando el déficit de pitchers en la Serie Nacional y por consecuencia el espectáculo ha perdido calidad. Cuando se hace alusión a la falta de pitchers el principal agravante que se aborda es la falta de control, en el béisbol la fama y fortuna de un pitcher dependen de su maestría en la zona de strike. Como se señaló anteriormente, los sistemas ya desarrollados han constribuido sustancialmente al desarrollo tanto de pitchers como de árbitros. En Cuba no se dispone de dichos sistemas, ya sea por impedimentos económicos y/o políticos, es por ello que se ha emprendido una investigación que pretende comprobar si es posible diseñar e implementar un sistema de bajo costo para determinar en tiempo real y con un error de precisión cercano al de los sistemas ya existentes si un lanzamiento de un pitcher fue strike o bola, utilizando una única cámara PlayStation Eye \cite{PlayStationEye} (ver Fig. \ref{fig:PlayStationEye}).

\begin{figure}[h!]
    \centering
    \includegraphics[width=5cm]{Graphics/PlayStation-Eye.png}
    \caption{Cámara Sony PlayStation Eye.}
    \label{fig:PlayStationEye}
\end{figure}

En el año 2010 la compañía Sony desarrolló el PlayStation Move \cite{PlayStationMove}, un sistema de control de videojuegos mediante sensores de movimiento. Como parte de este sistema se lanzó al mercado con un precio de 20\textit{usd} el dispositivo PlayStation Eye, hoy ese precio oscila alrededor de los 5\textit{usd}. Este dispositivo contiene una cámara capaz de capturar hasta 120 cuadros por segundo, a diferencia de las cámaras web regulares que solo captan 30 cuadros por segundo. Además, cuenta con un chipset de alta calidad comparable con los encontrados en muchas cámaras dedicadas a la visión por computadora. Esto combinado con su bajo costo la convierte en una gran opción a la hora de tomar la decisión de cuál dispositivo utilizar para desarrollar un sistema computacional basado en visión por computadora.

\subsection*{Objetivos}

Atendiendo al alto costo de los sistemas ya existentes y las ventajas que brindan los mismos, se propone como objetivo principal comprobar si es posible desarrollar un sistema que permita determinar en tiempo real si un lanzamiento de un pitcher de béisbol fue strike o bola mediante técnicas de visión por computadora utilizando una única cámara PlayStation Eye.

A continuación se enumeran los siguientes objetivos específicos:
\begin{itemize}
    \item {Estudiar técnicas para la calibración de la cámara.}
    \item {Probar técnicas de eliminación de ruido en tiempo real dada una secuencia de imágenes capturadas por la cámara.}
    \item {Determinar si en una imagen se encuentra el Home Plate.}
    \item {Ubicar los 5 puntos extremos del Home Plate en la imagen.}
    \item {Transformar la imagen brindada por la cámara a una vista top-view.}
    \item {Determinar si en una imagen se encuentra la pelota.}
    \item {Proyectar la trayectoria de la pelota en una secuencia de imágenes.}
    \item {Verificar si el lanzamiento fue strike o bola, así como determinar la velocidad del lanzamiento.}
    \item {Obtener datos estadísticos que permitan determinar la precisión y la velocidad de un pitcher después de un conjunto de lanzamientos.}
\end{itemize}

% \subsection*{Contribuciones}

% \section*{Organización de la Tesis}

% El capítulo ...