\chapter{Implementación}

En este capítulo se describe la estructura general del software y las herramientas utilizadas para la implementación del sistema. Además se exponen los resultados alcanzados a partir de la obtención de varias escenas de prueba.

\section{Estructura General}

% Antes de utilizar el dispositivo es necesario realizar la calibración. Este procedimiento
% se hace cada vez que cambia la posición relativa entre las cámaras, por ejemplo cuando se
% altera la línea base o ajusta la posición de alguna cámara en la estructura metálica. En el
% algoritmo 2 está el seudocódigo para generar el mapa de profundidad.
% En la línea 6 se crea un mapa que especifica la transformación de cada pixel en las
% imágenes. Este paso evita tener que calcular la rectificación en cada iteración. El método
% AdjustCameras(cam L , cam R ) modifica los parámetros de ganancia y exposición de las cáma-
% ras para que se adapten a la iluminación de la escena y desactiva los ajustes automáticos.
% Esto asegura que ambas cámaras mantengan los mismos parámetros. El ciclo en la línea 10,
% donde se capturan las imágenes y se genera el mapa de disparidad, se realiza en un hilo
% independiente.

\subsection{Herramientas}
