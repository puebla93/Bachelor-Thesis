\chapter*{Opinión del tutor}\label{chapter:supervisorOpinion}

El desarrollo de nuevas tecnologías de seguimiento junto a importantes avances en el campo de la visión computacional han generado una tendencia a aplicar medios digitales para enriquecer las interacciones en los deportes. Por ejemplo el sistema \textit{PITCHf/x} permite mejorar la experiencia de los televidentes a partir de visualizaciones de las jugadas, superponiendo en la transmisión en vivo del juego la trayectoria de los lanzamientos y la zona de strike. Otros tienen el objetivo de apoyar a los árbitros en la toma de decisiones, como el sistema "Ojo de Halcón" que avisa al árbitro de fútbol cuando el balón entra en la portería. Por el costo de implementación y complejidad de la puesta en práctica, estos sistemas están reservados para las ligas y clubes de mayor prestigio y solvencia económica. Incluso en países desarrollados no es común contar con estos sistemas en los centros deportivos de las escuelas primarias y secundarias a pesar de ser esta una etapa clave en el desarrollo de todo deportista.

Teniendo en cuenta el desarrollo de nuevas plataformas computacionales y cámaras de bajo costo con altas prestaciones, el objetivo principal del trabajo presentado por el estudiante José Javier Puebla Pérez es la implementación y validación de un sistema de detección y caracterización de lanzamientos de un pitcher de béisbol utilizando una cámara PlayStation Eye. En consecuencia desarrolló una investigación de las soluciones existentes donde cabe destacar la dificultad adicional que imponen estos sistemas debido a que son propietarios y de código cerrado. Realizó un análisis de las posibles posiciones de un sistema monocular teniendo en cuenta la viabilidad de cada una. Diseñó un algoritmo robusto para la detección y alineación del home plate e implementó un algoritmo para la estimación de la trayectoria y velocidad de la pelota a partir de un conjunto de posiciones observadas, con la complejidad adicional de que estas posiciones tienen valores atípicos debido al ruido introducido por la cámara y la escena. Para ello necesitó poner en práctica diversos conocimientos adquiridos durante la carrera así como de otros no comprendidos en el plan de estudio, demostrando que ha alcanzado el nivel y la madurez suficientes para obtener el título de Licenciado en Ciencias de la Computación.

\begin{flushright}
	Lic. David Alejandro Darias Torres\\
	Mayo, 2018
\end{flushright}
